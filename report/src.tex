\documentclass[10pt]{article}

% packages
\usepackage{amsmath}
\usepackage{float}
\usepackage{bm}
\usepackage{graphics}
\usepackage{epstopdf}
\usepackage{enumitem}
\usepackage{color}
\usepackage{minted}
\usepackage{cite}
\usepackage[hidelinks]{hyperref}
\usepackage[xindy, toc, acronym, nonumberlist]{glossaries}
\usepackage{titling}

% font settings
\usepackage[T1]{fontenc}
\usepackage[utf8]{inputenc}
\usepackage[english,french]{babel}
\usepackage[babel=true,kerning=true]{microtype}
%\usepackage[urw-garamond]{mathdesign}
%\usepackage{garamondx}
%\usepackage[scaled]{beramono}
%\usepackage{titlesec}
%\titleformat{\section}[hang]{\normalfont\scshape\filcenter}{\thesection.}{1em}{}
%\renewcommand{\thesection}{\Roman{section}}

% figures
\usepackage[subpreambles]{standalone}
\usepackage{tikz}
\usepackage{pgfplots}
\usetikzlibrary{arrows}

\definecolor{gray}{RGB}{155,155,155}

%\usepackage[sc]{mathpazo}
%\linespread{1.05}

\newminted{cpp}{bgcolor=gray!15,fontsize=\footnotesize,mathescape,frame=lines,framerule=0.3pt}


\include{glossary}
\makeglossaries

\setglossarystyle{long}

\DeclareMathOperator*{\argmax}{\arg\!\max}

\binoppenalty=10000
\relpenalty=10000

\sloppy

\begin{document}

\selectlanguage{english}

\title{
    Boosted Multi-block LBP Face Detection
}

\author{
    James Batten $\quad$ Valentin Iovene \\
    \texttt{batten\_j@epita.fr} $\quad$ \texttt{toogy@lrde.epita.fr}
}

\def\tightlist{}

\begin{titlingpage}
    \maketitle

    Face detection is the task of localizing faces in images. One must
    distinguish between face \emph{detection} and face \emph{recognition}. The
    latter is the task of recognizing someone from an image of her face. \\

    Face detection can be seen as a function $f$ taking an image as input and
    returning \emph{bounding boxes} around the faces contained in this image. \\

    \begin{center}
        \includestandalone[width=8cm]{figures/face-detection-diagram/figure}
    \end{center}

    Face detection

    \begin{itemize}

        \item is used by \emph{Facebook} to detect faces in images you share and
            make it easier for you to tag your friends,

        \item can be used to \textbf{track} faces in videos of people moving
            around (e.g. \emph{Snapchat}'s face swap feature requires face
            detection),

        \item and is a prerequisite for any real-life applications of face
            recognition \emph{(e.g. Law Enforcement localizing a suspect in a
            big city using video surveillance systems)}.

    \end{itemize}

    The detection process needs to be fast enough to keep up with the demand
    (millions of users or streamed images). \\

    There has been extensive research in this area but one of the major
    breakthrough happened in 2001 when \emph{Paul Viola} and \emph{Michael
    Jones} published their object detection framework \cite{viola} (which was
    named the "Viola-Jones framework").

    Even though it was invented more than 15 years ago \emph{(time of writing)},
    \emph{OpenCV}'s face detection implementation (source) is still based on it.
    \\

    This article explains how the \emph{Multi-Block Local Binary Patterns
    (MB-LBP)} visual descriptors (which will soon be introduced) can be used in
    lieu of the \emph{Haar-like features} used by the original
    \emph{Viola-Jones} framework for face detection. This approach was
    demonstrated by Zhang et al. in 2007. \cite{Liao2007}
\end{titlingpage}

\tableofcontents

\include{facedetect}
\nocite{*}

\printglossary
\clearpage
\printglossary[type=acronym,style=long]

\clearpage

\bibliographystyle{abbrv}
\bibliography{bibliography}{}

\end{document}
